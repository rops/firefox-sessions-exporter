%% LyX 2.0.2 created this file.  For more info, see http://www.lyx.org/.
%% Do not edit unless you really know what you are doing.
\documentclass[twocolumn,english]{IEEEtran}
\usepackage[T1]{fontenc}
\usepackage[latin9]{inputenc}
\usepackage{babel}
\usepackage{amsmath}
\usepackage{amssymb}
\usepackage[unicode=true,
 bookmarks=true,bookmarksnumbered=true,bookmarksopen=true,bookmarksopenlevel=1,
 breaklinks=false,pdfborder={0 0 0},backref=false,colorlinks=false]
 {hyperref}
\hypersetup{pdftitle={Your Title},
 pdfauthor={Your Name},
 pdfpagelayout=OneColumn,pdfnewwindow=true,pdfstartview=XYZ,plainpages=false}

\makeatletter

%%%%%%%%%%%%%%%%%%%%%%%%%%%%%% LyX specific LaTeX commands.
%% Because html converters don't know tabularnewline
\providecommand{\tabularnewline}{\\}

%%%%%%%%%%%%%%%%%%%%%%%%%%%%%% User specified LaTeX commands.
\usepackage{babel}
\usepackage{babel}
\usepackage{babel}
% for subfigures/subtables
\ifCLASSOPTIONcompsoc
\else
\fi



\providecommand{\theoremname}{Theorem}

\makeatother

\begin{document}

\title{Preservation and migration of browsing\\
 sessions in mobile environments}


\author{Eros Lever, Daniele Rossetti}


\IEEEspecialpapernotice{June 2012}
\maketitle
\begin{abstract}
Nowadays the web browsing activity on a mobile device may be influenced
by several factors which may create unwanted delays or unpredicted
behaviors. Connection drops are the most common obstacles for browsing
and they may occur for different reasons, e.g. external circumstances
or OS internal decisions as well. In our work we aim to provide a
solution that can allow the user activities to be protected in this
kind of situations and for this purpose we are introducing a browser
session preservation system that allows a user to take a snapshot
of an active web session state on a mobile browser. The system responds
to user requests and automatically reacts as well to external events
which may cause the corruption of the current session. The infrastructure
allows the user to retrieve the snapshot at a later time to recover
the same active web session, and possibly migrating it to a different
device. The design of the system is based on a browser-side plug-in
that can capture the browser session state. 
\end{abstract}

\section{Problem Definition}

Our work lives in a context of a smart mobile operating system (i.e.
MorphoneOS, Android-based OS\cite{Morphone}), that focuses its attention
on improving several aspects of a mobile device and adapts itself
to the surrounding environment, in order to provide a better user
experience by smartly facing situations than can affect the device
behaviour.\\
 \\
 The main problem we addressed was the fact that the system can not
guarantee by itself the preservation of the navigation session if
a connection drop occurs, causing the user to partially or completely
lose her activity.\\
 The connection drop may occur with external events as well as with
internal ones, and it can be predictable or unpredictable, here follows
a list of example of these possible scenarios.\\

\begin{enumerate}
\item \emph{Unpredictable External} \\
 Trivially, just think of a situation when the user is browsing a
web page, compiling a form and submitting it when unexpectedly the
data connection goes down, due to physical limitations such as distance
from the base-station or interferences, causing the web transaction
to be aborted and forcing the user to start again her operation. 
\item \emph{Predictable External}\\
 When the user enters a Faraday cage such as an elevator or an underground
parking loft, the connection will likely drop, but this event could
be predicted by a smart user, which can decide to delay her entrance
if she is performing a transaction. 
\item \emph{Unpredictable Internal}\\
 In these category lie situations such as hardware failure inside
the device or system crash; this kind of issues is practically impossible
to recover from, without implying a restart from a known working situation. 
\item \emph{Predictable} Internal\\
 When the system itself, due to some internal choices, causes the
connection to go down or to switch from a data network to another
(e.g. switching from 3G connection to EDGE connection) to reduce the
power consumption. As a matter of facts, rather than trying to improve
the technology beyond the battery autonomy, which could be a way too
tough job, the operating system may decide to pay more attention on
using its available resources more efficiently. One way to handle
it is to dynamically switching from a faster but more power wasteful
data network (e.g. 3G or Wireless ) to a slower but more power efficient
one (e.g. EDGE) in some particular situations, such as when running
out of power. Although it seems a quite simple mechanism, the system
must guarantee that this power-saving switching behavior is completely
transparent to the user, so that she can enjoy the benefits of a more
power-conscious device without having her activity compromised. Unfortunately
switching network will cause the connection to be stalled for a small
timeframe and, as pointed out in the first case, it will likely corrupts
the user web transaction. 
\end{enumerate}
In addition to these problematic situations, we also aimed to provide
a solution for \emph{another kind of scenario}, that is whenever the
user may decide to move her activity to a different device and continue
browsing without any additional effort. Unfortunately existing solutions
do not allow to completely migrate a browsing session in such a transparent
way that the user can resume her task like nothing changed. \\
 \\
Once stated this, our \emph{goal} is to guarantee a better user experience
to support the defined scenarios, allowing the user to restore and
preserve her browsing session as connection faults occur and to design
a smart infrastructure to support session migration such that no effort
are required to the user to complete the task.


\section{Analysis}

In this section we analyze definition, possible solutions and our
implementation of the given problem.


\subsection{Browser State Definition}

The first step was to identify the set of elements that describe the
state of the browser for the current session, keeping in mind that
a session may be recovered also in a device completely different from
the original one. We identified mainly two elements: \\
 
\begin{enumerate}
\item \emph{Cookies} \\
 As well-known, HTTP is a stateless protocol and exploits cookies
mechanism to keep the state of the current user's navigation. They
basically contains pieces of data represented by strings of characters
limited in size and duration that are used to track the same use from
a page to the following ones. Therefore cookies were surely components
to keep in consideration when saving a session state, especially when
we would need to migrate the session and continue browsing on a different
device\cite{Canfora2005}. \\
 
\item \emph{Current Visited Page} \\
 Trivially, we needed to send the user back to the page she was visiting.
Since we wanted to guarantee that after the session recovery, the
user would find herself exactly in the same state as before the restoring,
keeping track of the URL visited may not be sufficient. We will discuss
this topic later, since we took several ideas in considerations before
finding the most appropriate for the aspired behavior. 
\end{enumerate}

\subsection{Possible Solutions}

After defining the elements we would need to gather from the browser
to preserve the session state, we needed to find the best approach
to interact with the browser to actually obtain them. \\
 
\begin{enumerate}
\item A first hypothesis was to build up a separate application, make it
running in background and request the wanted information to the browser
when needed. This approach may seems to have one important advantage
that is compatibility, meaning the fact of being totally independent
from the browser application used on the device. Unfortunately, due
to the very different implementation of many browsers, it is impossible
to build up an application that can effectively interact with any
browser. It would have been necessary to design different interfaces
for every browser we would have wanted to support. On the other hand,
this was not the real drawback of this approach, since the number
of available browsers for Android OS is not so significant and we
could have focussed just on the two or three most popular. The reason
why we discarded this solution was for its true limit, namely the
fact that, mainly for security reason, a browser would not simply
allow any external application to retrieve its private components,
e.g. most of the browsers encrypt their cookies in order to prevent
the stealing of sensible datas. \\
 
\item On second thoughts, our choice moved to a different solution. We realized
that the best way to effectively interact with the browser core was
to build up a component that could extend its functionality, still
being inside the context of the browser itself, that is to say a browser
add-on. This choice had all the features required to retrieve all
the components we needed to have a consistent state of a session,
since it allowed us to interact directly with the browser and all
of its elements just from inside its core\cite{Song}. Although this
solution might seem to have lack of portability and compatibility,
since every browser needs different implementations for its extensions,
we thought that accurately choosing one of the most common and supported
browsers we could reach out to a wide target of devices as well. \\

\end{enumerate}
%\begin{lyxlist}{00.00.0000}
%\item [{\textit{Chosen~Platform}}]
\textit{Chosen~Platform} \\


Our target browser has been the popular Mozilla Firefox Mobile (codename:
Fennec / current version: 14.0)\cite{FirefoxMobile}. The development
of this application is significantly active and it is one of the fews,
also among the most commonly used, that supports extensions (e.g.
Google Chrome and Opera do not currently support add-ons in their
current mobile versions), providing one of the most complete and advanced
SDK for add-ons development. We also preferred Firefox since, according
to us, Mozilla is a more trusted source than others small open-source
projects and can guarantee a more stable and reliable solution. The
official version available on Google Play (Android official application
store ) supports just ARMv7 processors and requires at least Android
2.1. Anyways we have been able to find a parallel development project\cite{FirefoxARMv6}
which supports ARMv6 as well, while Mozilla is currently working to
officially support this family of processors as well. Thanks to that,
we could support almost all the devices running AndroidOS. %\end{lyxlist}



\subsection{Implementation}

We just want to outline in this section the most interesting and problematic
aspects of the development process.\\
 As already pointed out, we needed to get the cookies from the browser
and somehow preserve the current state of the last visited page, exploiting
the fact of being completely able to interact with the browser thanks
to the add-on design.\\



\subsubsection{Saving a snapshot of a page}

When we talk about the current state of the page, we refers to the
fact that pages are not static but they may contain some data input
by the user and not yet committed (e.g. a form being filled ) and
it also may have been generated dynamically by previous HTTP (POST/GET/)
synchronous or asynchronous (AJAX ) request. \\
 
\begin{itemize}
\item A very basic and simple idea was to just save the URL, but since,
again , we want to make the session restoring transparent to the user
as much as possible, it could not be sufficient : we would not had
the same state if the page were built dynamically or just if some
data have been input by the user. This is because nowadays pages are
not always static documents but they may vary from time to time or
even on a per access basis and it can be also altered with different
kinds of interaction, such as user input and active code (Javascript
or Flash). \\
 
\item A more appropriate solution could have been saving the URL and additionally
the input data, by parsing the HTML, and eventually facing some dynamic
generation of the page by tracking all the HTTP requests made by the
browser since the first page visited by the user. This solution would
undoubtedly been too computationally heavy for a mobile processor
and we did not want to add more stress in terms of power consumption
or system's slowing down to the device. \\
 
\item Alternatively, we moved on a more lightweight solution, which consists
in saving directly the whole HTML page to a local file as a real snapshot
of the current DOM, so that all the dynamically generated elements
and inputs would have been preserved. Afterwards, restoring the page
would have just meant reload it locally. \\
 
\begin{figure}
\caption{}
\end{figure}

\end{itemize}

\subsubsection{Problems of a local copy}

Having a local copy of the last visited page solved the problems reported
before, but on the other hand it also introduced a non-trivial issue.
Loading it locally would have meant that all the relatives links contained
in the page ( hrefs,styles,scripts ) would had just simply stopped
working, since the base URL would have become localhost rather than
the remotehost.\\
 % e.g. )  link: "/some_page.html" rather than refers to "www.somehost.com/some_page.html" it would have referred to "localhost/some_page.html"

\begin{itemize}
\item \emph{Replace the URLs}\\
 At first, we thought to solve this issue by replacing all the occurrences
of relative URLs by their correct absolute value:\\
 \\
 \textit{/some\_page.html $\Rightarrow$ www.somehost.com/some\_page.html}\\
 \\
 We exploited an open-source code from a web-sanitizer javascript
library and adapted to our needs to parse the whole document and fix
the (static) problematic URLs . Making some performance (ms/page)
tests we got these quite unsatisfying results, in terms of time required
to complete the task. In these tests we tracked the required time
of saving the session adopting the proposed methodology; the test
ran on known web pages allowing us to determine the amount of URLs
correctly replaced. Since also the size of the page influences the
requested time, we chose to have pages with three different weights.\\



\begin{table}[!ht]
\centering \caption{Performances}


\begin{tabular}{cccc}
Website  & RequiredTime  & Size  & URLsCorrectlyReplaced\tabularnewline
Twitter  & 800ms  & 50kb  & 7/7 \tabularnewline
Facebook  & 930ms  & 200kb  & 10/12 \tabularnewline
Amazon  & 1900ms  & 350kb  & 21/39 \tabularnewline
\end{tabular}
\end{table}



Although we could get good results in terms of required storage space,
the URLs replacements was not so satisfying. With more complex URLs
our system could not manage to fix all of them and in addition increasing
the page size led to a significant performance loss. \\


\item \emph{Rebase the URLs} \\
 The fact that it is practically impossible to recognize and alter
all the URLs, and that it required a high amount of computation, pushed
us to find some different solution that could led us to more acceptable
performances. We then spotted and took advantage of one of the brand
new HTML 5 features which is the <base> tag\cite{BaseTag} that actually
rebase ( i.e. set the base path ) almost all the relative paths contained
in the page. The only links that are not affected by the <base> tag
are the ones referred in the CSS sections of the page and <link> tags
that have a protocol relative ``HREF'' attribute. Addressing only
those elements of the DOM we could get a really quick fix by URL replacing,
without affecting the system performance. As expected, appending just
the \emph{base} tag and accomplishing a fast parse-and-replace on
a limited number of code lines, gave us more satisfying results.\\



\begin{table}[!ht]
\centering \caption{Performances}


\begin{tabular}{cccc}
Website  & RequiredTime  & Size  & URLsCorrectlyReplaced\tabularnewline
Twitter  & 300ms  & 50kb  & 7/7 \tabularnewline
Facebook  & 330ms  & 200kb  & 12/12 \tabularnewline
Amazon  & 900ms  & 350kb  & 39/39 \tabularnewline
\end{tabular}
\end{table}



Since we knew in advance which kind of URLs needed to be replaced,
we could avoid to parse all the document, just address them in the
DOM tree and get more brilliant results.\\


\item \emph{Input data}\cite{Hsieh2006}\\
 Another problem we encountered was the fact that Firefox (even in
the Desktop edition) does not dump by default the content of user
edited form fields. There are basically two methods to cope with this
issue, that are tracking all the user clicks and typed keys, or taking
the content of the form the exact moment the page is being saved.
Since the first way introduces unneeded power consumption and may
interfere with user privacy, having she thinking that something is
tracking all her input, we decided to adopt the second methodology.\\
 When saving the session, the content of the input forms is made effective
transcribing its value with the \emph{setAttibute} function in Javascript.\\
 
\end{itemize}

\subsubsection{Effectively responds to connection loss}

We thought of different possible solutions that could help preventing
such a situation, namely a first one having a timer that regularly
saves the session and a second one that detects connection drops and
handles them saving the session before it gets lost. \\
 
\begin{itemize}
\item We first decided to use a timer, but experimentally we saw this solution
not being truly effective. The reasons around this impreciseness were
the trade off between saving interval and power efficiency, more in
the detail: to avoid losing last performed actions during the session,
it has to be saved frequently, obviously causing unwanted power consumption.
Although there are a few improvement that can be took in account,
such as avoiding saving multiple times the very same session and other
more complicated as altering only the differences from the previous
state when saving the same web page repeatedly, these proposed enhancements
still do not really provide a valid solution since they requires the
session to be saved frequently maybe adding costs in computing and
tracing differences. \\
 
\item Therefore we decided to take advantage of Firefox Mobile built-in
notifications and observers system, mainly the ``network:offline-about-to-go-offline''\cite{MDNObserver}
that detects connection drops and notifies our extension before altering
the active page. This way we are able to save the session only when
it is truly necessary, avoiding wastes in power consumption. \\
 
\end{itemize}

\subsubsection{Session packing}

We distinguished between HTML document and cookies using two different
files, one each. Cookies are saved encoded in JSON format, which is
handy to use in Javascript, as they are retrieved from the browser
APIs (we will discuss the security limits of this approach in \ref{sub:Security}).
The obtained HTML is then combined with the cookies, and these two
items are saved in a ZIP file to shrink the size and avoid wastes
in term of memory footprint on the device and transfer duration when
sent over the network to another device. The resulted ZIP file is
then stored in a local directory sessions on the SD card, making it
accessible by third-party applications, such as SwishApp\cite{SwishApp},
which allows to transfer datas between near devices. The interaction
between our extension and the SwishApp is detailed in section ~\ref{testswish}.\\



\subsubsection{Session restoring}

Being able to restore the session means being able to bring back the
user to the very same web page she was interacting with, allowing
her to continue her navigation using the cookies that were saved.
Our extension provides a restore functionality triggerable by the
user, that once invoked (through a resource within the extension,
accessed as a \emph{chrome://} registered path) it will unpack the
session from the ZIP file provided, update the cookies with their
saved value, and then show the HTML document.


\section{Test Case}

In this section we describe three scenarios used by our tests to point
out different aspects of session handling.


\subsection{Simulating connection loss}

To test the success of our system with respect to unexpected connection
losses we design a very simple Android background application that
randomly turn off the Wireless connection. Thanks to the reliable
system of Android and Firefox in detecting the network going down,
we could manage to handle successfully every connection faults, correctly
saving and restoring the navigation session.


\subsection{Forced Sessions Destruction}

\label{testdrop}

To test if our system actually managed to successfully and completely
recover from scratch a session we added a debug functionality to our
add-on that actually destroys all the cookies saved by the browser.
A simple test consisted in logging in a website, fill in some form,
saving the session and then destroying all the cookies (meaning that
the website can not validate the logged user ). After restoring the
user was still logged in with the input data in the right place, proving
that the session was correctly recovered.


\subsection{Session Migration with SwishApp}

\label{testswish}

To test the adaptativity of our system with respect to the session
migration we exploited the Swish application to move a session from
a device to another. In order to support it we had to build an helper
application that could manage the session file as soon as the second
device receive it, since the SwishApp just opens the received file
with the default handler. More precisely the helper application registers
itself as the default handler for .session file ( custom extension
we defined for our session files ), just moves the received file to
the sessions directory and opens Firefox with a special URL, calling
back the add-ons functions that start the restoring process.Once the
session is moved, the system managed to successfully recover the session,
as already proved with ~\ref{testdrop} test case.


\section{Limits and Possible Improvements}

Even if our work provides a working solution, there can be few limitation
that could be taken into account for possible future works.


\subsection{Client-side issues}

Our current solution does not take care of the running state of client-side
scripts, such as Javascript, meaning that the values of the variables
will be lost during the session saving/restoring process. As a practical
example, if a page contains a counter, dynamically updated by Javascript,
the saving system can not access the internal state of the script
so, after restoring, the counter will be reset to the initial value.
Similarly, Flash scripts state would not be recovered, meaning that
for example watching a video on Youtube and then restoring the activity
would not take the user the right timing of the video. Another problem
is that switching the context of the script from remote to local may
lead to some security issues, since when executed in remote pages
Flash objects may have different privileges with respect to when they
are executed inside a local page.


\subsection{Multimedia streams}

When dealing with streams, such as audio or video media, it could
be useful to restart from a specific position. To detect this specific
position is really an hard task to accomplish, since the methodology
may vary from case to case. Moreover technologically speaking, it
is not always possible to ask a server for only a part of a stream.
These limits prevented us to cope with multimedia streams.


\subsection{Security\label{sub:Security}}

Migrating cookies to another device may lead to security issues, for
example migrating an authenticated session to a device under control
of another person, can potentially introduce an abuse of the cookies
performing operations as if the receiving person were the other party.
This is intrinsic to cookies and impossible to prevent without modifying
the authentication check server-side, adding controls such as IP address
or User-Agent in use.


\subsection{Mobile-vs-Desktop pages layout}

When migrating a session from a device to another, it may be the case
in which one should see the mobile version of a webpage and the other
the desktop one. When dealing with bare links, it is normally just
matter of add a ``m.'' in front of the remote hostname, but since
this is not a standard we can not assume this methodology to be always
valid. Furthermore, using two different hostnames may lead to invalidate
the HTTP cookies, preventing the user the ability to continue her
browsing session.


\appendices{}

\bibliographystyle{IEEEtran} \phantomsection\addcontentsline{toc}{section}{\refname}
  \bibliographystyle{IEEEtran}
\nocite{*}
\bibliography{biblio}

\end{document}
